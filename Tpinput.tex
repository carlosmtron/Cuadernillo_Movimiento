\subsection*{Actividad 1}
\subsubsection{Trabajo Práctico N\textordmasculine 1}


\textbf{Título:} Movimiento Rectilíneo

\textbf{Objetivo:} Describir el movimiento rectilíneo de una pequeña esfera que cae por el interior de un recipiente que contiene detergente.

\textbf{Materiales:} probeta, detergente, esferas metálicas, cronómetro,  papel milimetrado, cinta métrica.

\textbf{Procedimiento:}
\begin{itemize}
  \item Deja caer la esfera en el detergente y anticipa el tipo de movimiento que posee, según lo estudiado en cinemática.
  \item Observa que a lo largo del tubo se han hecho marcas cada 5\,cm.
  \item Planifica cómo medir el intervalo de tiempo que demora la esfera en recorrer las distancias marcadas en el papel milimetrado.
  \item Discute los procedimientos para la elección del sistema de referencia.
    \item Discute cómo reducir las incertezas que afectan a las mediciones.
\end{itemize}

\noindent
\textit{Primera Parte:}
\begin{itemize}
  \item Deja caer la esfera dentro de la probeta. Mientras la esfera desciende, mide para cada  distancia $\Delta x$ el tiempo $\Delta t$ que demora en recorrerla.
  
  \item Calcula la velocidad media  $v_{m,i}$ para cada intervalo y analiza los valores obtenidos.
  
  \item Completa la siguiente tabla:
  
  \begin{table}[!htb]
    \centering
    \label{tab:vm}
    \begin{tabular}{|c|c|c|c|c|}
    \hline
    N$^{\circ}$ & $\Delta\,x_i$ [cm] & $\Delta\,t_i$ [s] & $v_{m,i}$ [cm/s]                        & $\varepsilon \ = \ |v_{m,prom} - v_i|\si{[cm/s]}$ \\ \hline
    1           &                  &                 &                                         &                               \\ \hline
    2           &                  &                 &                                         &                               \\ \hline
    3           &                  &                 &                                         &                               \\ \hline
    4           &                  &                 &                                         &                               \\ \hline
    5           &                  &                 &                                         &                               \\ \hline
                &                  &                 & $v_{m,prom} \ = \ \frac{\sum \ v_{m,i}}{N}$ &                               \\ \hline
    \end{tabular}
    \caption{Tabla a construir en la primer parte del trabajo.}
  \end{table}
  \item Expresa correctamente el resultado de la velocidad media como $v_m \ = \ v_{m,prom} \ \pm \ \varepsilon_{max}$ (indicando unidad, dirección y sentido).
\end{itemize}

\noindent
\textit{Segunda Parte:}
\begin{itemize}
  \item Deja caer la esfera y mide el tiempo que transcurre desde el origen hasta que pasa por cada una de las posiciones marcadas.
  \item Mide las posiciones correspondientes a cada una de las marcas hechas sobre el tubo desde $x_0 \ = \ 0$\,cm.
  \item Completa la siguiente tabla:
  \begin{table}[!htb]
    \centering
    \label{tab:xvst}
    \begin{tabular}{|c|c|}
    \hline
    $x$\,[cm]                & $t$\,[s]                \\ \hline
                           &                       \\ \hline
                           &                       \\ \hline
                           &                       \\ \hline
                           &                       \\ \hline
                           &                       \\ \hline
    \multicolumn{1}{|l|}{} & \multicolumn{1}{l|}{} \\ \hline
    \end{tabular}
    \caption{Tabla a construir en la segunda parte del trabajo.}
  \end{table}
  \item Grafica $x \ = \ x(t)$.
 
  \end{itemize}


   
  Al finalizar la experimentación, confecciona un informe que contenga: título, objetivo, materiales  (indicando características de los instrumentos usados), procedimiento, observaciones, incluyendo tablas, gráficos, análisis de los resultados obtenidos y sus incertezas, y conclusiones.

  Recuerda que las conclusiones deben responder al objetivo del Trabajo Práctico, lo que en este caso implica describir todas las caracterísiticas observadas y medidas del movimiento de la bolita en el detergente. Puedes ayudarte respondiendo las siguientes preguntas: {\it ¿Se pueden aplicar algunos de los modelos estudiados en este capítulo para describir lo que ocurre con la bolita? ¿Cuáles? ¿Qué magnitudes físicas asociadas a esos modelos se pudieron identificar y medir? ¿Qué se podría decir sobre la calidad de las mediciones obtenidas? Con el equipamiento utilizado, ¿se podrían hacer mejoras al experimento? ¿Cuáles?}
